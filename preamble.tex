% LaTeX template, preamble file
% ------------------------------------------------------------------------------
% This is the preamble file for the project, in which package dependencies and
% custom commands should be defined.


% Base packages ----------------------------------------------------------------
\usepackage{verbatim}
\usepackage[T1]{fontenc}     % adds support for printing accented characters
\usepackage[utf8]{inputenc}  % adds support for inputting accented characters
\usepackage[british]{babel}  % set document language
\usepackage[square]{natbib}  % set citation style to use square brackets
\usepackage{hyperref}        % insert clickable hyperlinks
\usepackage{lastpage}        % reference last page (used to count numbered pages)


% Math packages ----------------------------------------------------------------

\usepackage{mathtools}  % math commands and environments
\usepackage{amssymb}    % additional math symbols
\usepackage{bbm}        % improved 'blackboard bold' styling
\usepackage{siunitx}    % si-units
\usepackage{mathrsfs}   % math letterlike symbols
\usepackage{stackengine}% circle around asteriks
\usepackage{physics}    % math operators

% Figure packages --------------------------------------------------------------

\usepackage[dvipsnames]{xcolor}    % font colouring functionality
\usepackage{graphicx}              % handling of images
\usepackage{caption}
\usepackage{subcaption}
\usepackage{booktabs}              % improved tables
\usepackage{array}                 % setting rowheight
\usepackage{tikz}                  % drawing vector graphics
\usepackage{circuitikz}            % drawing circuits
\usepackage{algorithm}             % algorithm floats
\usepackage[noend]{algpseudocode}  % pseudocode listings
\usepackage{listings}              % source code listings
\usepackage{enumitem}              % naming list items
\usepackage{pgfplots}
\usepackage{pgfplotstable}
\usepackage{wrapfig}

% Styling packages -------------------------------------------------------------

\usepackage{microtype}             % improve full justification and font kerning
\usepackage{emptypage}             % suppress page numbering on empty pages
\usepackage{fancyhdr}              % easy customisation of headers and footers
\usepackage{titlesec}              % easy customisation of section titles
\usepackage[margin=3cm]{geometry}  % adjust page margins
\usepackage[toc,page]{appendix}    % improved appendix control
\usepackage{framed}                % environments package
\usepackage[framed,amsmath,thmmarks]{ntheorem} % environments package
\usepackage{icomma}

% Font packages ----------------------------------------------------------------
% With pdflatex, fonts are limited to those provided in packages (as opposed to
% xelatex or lualatex, which can use system fonts).  The typeface is changed by
% simply importing the package for it.
%
% See
% * http://www.tug.org/pracjourn/2006-1/hartke/hartke.pdf
% * http://www.tug.dk/FontCatalogue/

\usepackage{lmodern}  % improved default font
\usepackage{pifont}

%todo
\usepackage[
%  disable, %turn off todonotes
  colorinlistoftodos, %enable a coloured square in the list of todos
  textwidth=\marginparwidth, %set the width of the todonotes
  textsize=scriptsize, %size of the text in the todonotes
  ]{todonotes}
% Customisations ---------------------------------------------------------------
\mathtoolsset{showonlyrefs, showmanualtags} %removes equation numbering
\usetikzlibrary{arrows.meta}
\usetikzlibrary{positioning,calc}
\usetikzlibrary{intersections}
\usetikzlibrary{shapes.misc}
\tikzset{>={Latex[width=1mm,length=1.2mm]},
  }   % arrow size in Tikz
%\usetikzlibrary{external}
%\tikzexternalize %activate!
% \tikzset{external/force remake} % indkommentér for at lave tikz figurer på ny

% Theorem enviroments
\theoremheaderfont{\normalfont\bfseries}
\theorembodyfont{\normalfont}
\theoremstyle{break}
\def\theoremframecommand{{\color{orange!80}\vrule width 5pt \hspace{5pt}}}
\newshadedtheorem{example}{Example}[chapter]

\def\theoremframecommand{{\color{blue!50}\vrule width 5pt \hspace{5pt}}}
\newshadedtheorem{definition}{Definition}[chapter]

\def\theoremframecommand{{\color{black!100}\vrule width 5pt \hspace{5pt}}}
\newshadedtheorem{theorem}{Theorem}[chapter]

\def\theoremframecommand{{\color{gray!100}\vrule width 5pt \hspace{5pt}}}
\newshadedtheorem{proof}{Proof}[chapter]

% Sets
\newcommand{\N}{\mathbb{N}}  % natural numbers
\newcommand{\Z}{\mathbb{Z}}  % integers
\newcommand{\Q}{\mathbb{Q}}  % rational numbers
\newcommand{\R}{\mathbb{R}}  % real numbers
\newcommand{\C}{\mathbb{C}}  % complex numbers
\newcommand{\I}{\mathbb{I}}  % imaginary numbers
\renewcommand{\L}{\mathcal{L}}
\newcommand{\U}{\mathcal{U}}
\newcommand{\Y}{\mathcal{Y}}
\newcommand{\B}{\mathcal{B}}

% Semantics
\let\times=\cdot
\newcommand{\euler}{e}%\mathfrak{e}}
\renewcommand{\exp}[1]{\euler^{#1}}
\newcommand{\amatrix}[2]{\left[\begin{array}{@{}*{#1}{c}|c@{}}#2\end{array}\right]}
\newcommand{\transpose}{^\intercal}
\newcommand{\directsum}{\oplus}
\newcommand{\cardinality}[1]{\mathit{card}\bigl({#1}\bigr)}
\newcommand\oast{\stackMath\mathbin{\stackinset{c}{0ex}{c}{0ex}{\ast}{\bigcirc}}}

% Other operators and concepts
\renewcommand\qedsymbol{$\blacksquare$}                                     % use black square as QED
\DeclarePairedDelimiterXPP{\laplace}[1]{\mathscr{L}}{\{}{\}}{}{#1}          % Laplace transform
\DeclarePairedDelimiterXPP{\invlaplace}[1]{\mathscr{L}^{-1}}{\{}{\}}{}{#1}  % Inverse laplace transform
\DeclarePairedDelimiterXPP{\fourier}[1]{\mathscr{F}}{\{}{\}}{}{#1}          % Fourier transform
\DeclarePairedDelimiterXPP{\invfourier}[1]{\mathscr{F}^{-1}}{\{}{\}}{}{#1}  % Inverse fourier transform

\DeclarePairedDelimiter\ceil{\lceil}{\rceil}
\DeclarePairedDelimiter\floor{\lfloor}{\rfloor}
\DeclarePairedDelimiter\norm{\lVert}{\rVert}


% Colours
\definecolor{aaublue}{RGB}{33,26,82}
\definecolor{verynice}{RGB}{106,126,37}
\definecolor{wind}{RGB}{20,80,135}


% Headers and footers
\fancyhead{}                      % clear default header fields
\fancyhead[RO]{\projectgroup}  % left on even, right on odd
\fancyhead[RE]{\leftmark}         % chapter name
\fancyhead[LO]{\rightmark}        % sections name
\fancyfoot{}                      % clear default footer fields
\fancyfoot[CE,CO]{\thepage}       % centered page numbers
\pagestyle{fancy}                 % activate fancy style
% \setlength{\headheight}{13.6pt}
% Chapter titles
% http://mirrors.dotsrc.org/ctan/macros/latex/contrib/titlesec/titlesec.pdf
\titleformat
{\chapter}
[hang]
{\Huge\bfseries}
{\thechapter\enspace\textcolor{aaublue}{|}\enspace}
{0pt}
{\Huge\bfseries}

% Predefined TikZ node styles
\tikzstyle{point} = [fill,shape=circle,minimum size=3pt,inner sep=0pt]
\tikzstyle{edge} = [fill=white,midway,inner sep=1pt]
% Listing style
\lstdefinestyle{custompy}{
  belowcaptionskip=\baselineskip,
  breaklines=true,
  language=Python,
  showstringspaces=false,
  basicstyle=\small\ttfamily,
  keywordstyle=\bfseries\color{green!40!black},
  commentstyle=\itshape\color{purple!40!black},
  identifierstyle=\color{blue},
  stringstyle=\color{orange},
}
\lstset{language=Python,style=custompy,captionpos=b}
\pgfplotsset{compat=1.15}